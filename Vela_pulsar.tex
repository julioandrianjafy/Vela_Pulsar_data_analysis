\documentclass[12pt]{article}

%Load packages
\usepackage{multicol}
\usepackage{graphicx}
\usepackage{subcaption}
\usepackage[margin = 1in]{geometry}
\usepackage{wrapfig}
\usepackage[font = small, labelfont = bf]{caption}

\title{Vela Pulsar with Ooty Radio Telescope}
\author{Andrianjafy Julio Cesar \\ University of Mauritius}


\begin{document}
\maketitle
\paragraph{}
This report briefly describes the data analysis that I have done on the observations of the Vela Pulsar with Ooty Radio Telescope. Ooty performed the observations centred on 326.5 MHz with a bandwidth of 16 MHz. The raw voltage data consists of 1 s obseving time from the Northern and Southern half of the telescope. The time interval between each voltage is 1/33 microseconds.
\begin{multicols}{2}
\section{Raw voltage}

By taking 100,000 random samples from the raw data, we show in Figure 1  that the voltage distribution follows a gaussian shape as expected. The mean value of the voltage from the northern feed, however, deviates from zero due to small bias in the telescope back-end.

\section{Dynamic spectrum}
The dynamic spectrum was obtained the same way as in Kishalay De report as follow:
\begin{itemize}
	\item apply Fast Fourier Transform to 512 points at a time; 
	\item compute the modulus square for half of the spectrum, corresponding to 256 frequency channels and a frequency resolution of about 64 KHz;
	\item average over 60 sets in time axis to get a time resolution of 1 ms.	
	
\end{itemize}

The signals were from the two halves of the array were then, added incoherently to improve the Signal to Noise Ratio (SNR). Figure 2 the correlated signal. We can observe 11 dispersed pulses from the vela pulsar within the 1 s observations. 

\section{Pulse properties}
The channels width was increased to 257 KHz for a better SNR. Assuming a gaussian shape, we fit a single pulse from 3 different channels and estimate their arrival time. The time delay of the pulse between two frequency $f_1$ and $f_2$ , with $f_1 > f_2$,  is given by: 
\begin{equation}
\Delta t \approx 4.15 \;(f_2^{-2} - f_1^{-2})\; \mbox{DM} \;\; \mbox{ms}
\end{equation}

where DM is the dispersion measure in (pc/cc). Equation 1 holds when $f_1$ and $f_2$ are in units of GHz. A single line $\Delta t = \mbox{DM}\delta f^{-2}$   was fitted with the data to calculate the DM where $\delta f^{-2}= 4.15 \;(f_2^{-2} - f_1^{-2})$. With the best fit of the line, the estimated DM is about 74.10 $\pm$ 10.2 pc/cc. The dedispersed dynamic spetrum and time series are shown in Figure 4. The pulsar period is calculated by applying a linear fit to the arrival times of each pulses in the dedispersed time series (see Figure 5), resulting into a period $\mbox{P} = 94.53 \pm 0.13 \; \mbox{ms}$. 
     
\end{multicols}

\newpage
\vspace{0.2in}
\begin{figure}[h]
  \centering
  \begin{subfigure}[b]{0.49\textwidth}
  	\centering
    \includegraphics[width=0.9\textwidth]{NorthernV.png}
    \label{fig:vnorth}
  \end{subfigure}
  \begin{subfigure}[b]{0.49\textwidth}
  	\centering
    \includegraphics[width=0.9\textwidth]{SouthernV.png}
    \label{fig:vsouth}
  \end{subfigure}
  \label{fig:voltage}
  \caption{Voltage distributions of 100,000 randomly selected samples for both the northern (left) and southern (right) feed.}
\end{figure}

\begin{figure}[ht]
    \includegraphics[width=.9\linewidth, height=0.24\paperheight]{correlated_sig.png} 
  	\label{fig:dscorr}  
  	\caption{Dynamic spectrum of the correlated signal. The frequency resolution is 64 KHz with a time resolution of 1 ms}
\end{figure}

\newpage
\begin{figure}
\centering
  \begin{subfigure}[h]{0.49\textwidth}
  	\centering
    \includegraphics[width=0.9\textwidth]{singlep0.png}
    \caption{}
  \end{subfigure}
  \begin{subfigure}[h]{0.49\textwidth}
  	\centering
    \includegraphics[width=0.9\textwidth]{singlep1.png}
    \caption{}
  \end{subfigure}
   \begin{subfigure}[h]{0.49\textwidth}
  	\centering
    \includegraphics[width=0.9\textwidth]{singlep2.png}
    \caption{}
  \end{subfigure}
   \begin{subfigure}[h]{0.49\textwidth}
  	\centering
    \includegraphics[width=0.9\textwidth]{linedt.png} 
    \caption{}
  \end{subfigure}
  \caption{The single pulse and best fit gaussian at 328.04, 326.24, 324.69 MHz are shown in figure a, b and c. The panel is showing the best line fit to estimate the dispersion measure.}
\end{figure}

\begin{figure}
\centering
  \begin{subfigure}[h]{1\textwidth}
  	\centering
    \includegraphics[width=0.9\textwidth]{dedisp.png}
  \end{subfigure}
  \begin{subfigure}[h]{1\textwidth}
  	\centering
    \includegraphics[width=0.9\textwidth]{dedtimes.png}
  \end{subfigure}
  \caption{Top: Dynamic spectrum after a dispersive delay correction with $\mbox{DM} $ value of $ 74.1 \pm 10.2$ pc/cc. Bottom: The total power of the signal summed across the frequency channels.} 
\end{figure}

\newpage
\begin{figure}
\centering
\includegraphics[width=1\linewidth, height = 0.5\paperheight]{period.png}
\caption{Linear fit to the pulse arrival times to estimate the rotation period.}
\end{figure}



\end{document}
